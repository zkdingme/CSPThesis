% 设置 biblatex 额外选项
% \PassOptionsToPackage{gbpub=false, gbtype=false}{biblatex}

\documentclass[degree=course, language=chinese]{sjtuthesis}
% 选项
%   degree=[doctor|master|bachelor|course],     % 必选,学位类型
%   language=[chinese|english],                 % 可选(默认:chinese),论文的主要语言
%   bibstyle=[gb7714-2015|gb7714-2015ay|ieee],  % 可选(默认:gb7714-2015),参考文献样式
%   review,                                     % 可选(默认:关闭),盲审模式

% 所有其它可能用到的包都统一放到这里了,可以根据自己的实际添加或者删除。
\usepackage{sjtuthesis}
\usepackage{float}
\usepackage{amsmath}

% 定义图片文件目录与扩展名
\graphicspath{{figure/}}
\DeclareGraphicsExtensions{.pdf,.eps,.png,.jpg,.jpeg}

% 导入参考文献数据库
\addbibresource{bib/thesis.bib}

% 信息录入,必须在导言区进行!
% !TEX root = ../thesis.tex

%TC:ignore

\title{大数据场景下的语言虚拟机优化调研} 
\author{丁贞锴   119037910044}
\studentid{陈一雄   119037910003}
\supervisor{陈悦    119037910017}
\department{齐明   119037930071}


%TC:endignore


% 自定义项目标签名称
% \sjtuSetLabel{
%   listfigure = {图\quad 录},
%   listtable  = {表\quad 录}
% }

\begin{document}

% 无编号内容:中英文论文封面、授权页
\maketitle

% 使用罗马数字对前言编号
\frontmatter

% 摘要
% !TEX root = ../thesis.tex

\begin{abstract}
  TODO
\end{abstract}


% 目录、插图目录、表格目录
\tableofcontents
\listoffigures
\listoftables
\listofalgorithms


% 使用阿拉伯数字对正文编号
\mainmatter

% 正文内容
% !TEX root = ../thesis.tex

\chapter{介绍}
过去十年见证了大数据的爆炸式发展,人类社会以前所未有的规模和速度创建大量数据。第二次工业革命以后,数据量越每10年翻一番;
从20世纪50年中期开启的信息革命带来的巨大生产力发展,导致人类社会从工业化时代进入信息化时代,新数据的产生进一步加速,约每三年翻一番。
前谷歌首席执行官埃里克·施密特在Techonomy'10会议上分享,现在短短2天产生的数据量比2003年之前整个人类社会产生的数据总量还多。
据国际数据公司(IDC)发布的2017年大数据白皮书预测,2025年全球大数据规模将增长至163ZB,相当于2016年的10倍。
大量数据的可利用性催生了大规模的数据密集型应用。大数据分析迅速成为一门重要研究,在金融、商业、医疗保健和政府管理等重要领域发挥作用,进而服务整个社会。
例如电商公司和广告公司可以对用户历史数据进行分析,来给用户提供个性化的产品;银行等金融机构的风控部门可以对每天的交易数据进行分析,打击金融诈骗等违法犯罪行为;
政府部门可以分析一段时间的人流数据,决定新的地铁站最合适的位置。

随着数字世界的不断扩展和大数据分析技术在社会生活中日益增长的重要性,公司、政府机构和学术机构对可快速运行大数据分析程序分析大规模数据的可扩展系统的需求也日益增长。
解决可扩展性挑战的主流方法是使用大量计算节点来进行分布式处理。常用的大数据处理系统包括Hadoop、Spark、Hive、DryadLinQ、Flink等。这些大数据处理系统大多是使用Java、
C\#或者Scala等面向对象的托管式语言开发的,因为托管式语言提供了快速开发的特性:自动内存管理和丰富的社区支持。但是使用托管式语言开发的程序必须要运行在托管式语言提供的
运行环境(runtime)或者更具体一点是托管式语言提供的虚拟机上。但是这些语言虚拟机在最初设计时并没有考虑到大数据的特性:数据量巨大和数据对象生命周期的epoch特性\cite{nguyen2016yak}(epoch特性是指一组对象在同一时间被创建,
在经历一段较长的程序执行时间后可以同时被回收), 因此大数据分析程序运行时,runtime
会带来巨大的性能开销,例如runtime内存管理中垃圾回收(GC)常常会占用40\% ~ 60\%的执行时间。

托管式运行时带来的如此大的性能开销是不可以接受的,那我们是不是可以抛弃托管式语言,回到非托管式语言?答案是否定的。使用非托管式语言例如C或C++进行开发确实可以避免托管式语言带来的
内存管理开销,但这也丧失托管式语言带来的便利:快速开发。众所周知,使用非托管式语言进行开发需要自己做内存管理等一系列工作,开发者需要写的代码更多,相应的,出错的概率也就越大。并且,
以非托管式语言调试内存错误是一项很痛苦的任务,尤其是在大数据处理系统数据量大、分布式执行环境和运行时间较长的特性下,调试的工作量更大。同时,由于大多数现有的大数据处理框架已经以
托管式语言进行开发,全部推倒以非托管式语言重构工作量太大,基本是不现实的。

因此,主流的研究方向还是在对现有的托管式runtime,主要是语言虚拟机进行优化。







% !TEX root = ../thesis.tex

\chapter{相关工作}
大数据的发展和基于托管式语言的平台开发带动了Hadoop生态圈的成功。目前Hadoop生态圈共有MapReduce,Tez,Spark及Flink等分布式计算引擎,
分布式计算引擎项目之间的竞争也相当激烈。随着各个项目的发展与日益成熟,通过改进分布式计算框架本身大幅提高性能的机会越来越少。
同时,在当前数据中心的硬件配置中,采用了越来越多更先进的IO设备,例如SSD存储,10G甚至是40Gbps网络,IO带宽的提升非常明显,
许多计算密集类型的工作负载的瓶颈已经取决于底层硬件系统的吞吐量,而不是传统上人们认为的IO带宽,而CPU和内存的利用效率,
则很大程度上决定了底层硬件系统的吞吐量。所以越来越多的项目将眼光投向了高级语言虚拟机本身,希望通过解决虚拟机本身带来的一些问题,
提高分布式系统的性能或是健壮性。大数据场景下,GC常常成为性能瓶颈,节点见的数据传输会带来巨大的性能开销,除此之外,多节点各自独立工作不合作也会带来重复劳动和性能损失。

\section{GC}
托管式语言提供了自动内存管理的特性,自动内存管理是通过GC实现的。在托管式语言实现的大数据系统中,
所有的数据都是存储在堆上的对象,对象的分配与回收都是由内存管理模块实现的,区别于非托管式语言像C或C++需要手动去malloc和free。
这固然减轻了开发者的工作压力,但是也带来了额外的性能开销。尤其是在大数据场景中,内存管理模块的对象分配与回收策略在最初设计时
并没有考虑到大数据的特征:数据量巨大并且大量数据具有相似的生命周期。这导致内存管理开销巨大,占据了大数据系统50\% 以上的执行时间,
极大的损害了性能,并且这种开销是不可以通过扩展的方式来解决。

考虑到数据对象数量巨大给内存管理带来的压力,FACADE\cite{nguyen2015facade}在编译层限制数据对象的数量,以此降低内存管理开销。FACADE编译大数据应用源码,
生成数据管理高效的代码。在生成的代码中,每个线程在堆中创建的对象的数量是有上限的,因此整个堆中数据对象相较于之前大大减少,
内存管理开销也相应降低。FACADE减少了大数据系统3\%-48\%的执行时间,将GC次数降低了88倍,内存消耗降低了50\%,但是编译前需要开发者在代码中标记数据对象,使用成本较高。

很多学者使用基于region的内存管理来降低GC开销。考虑到大数据中大量数据对象具有相似生命周期,Broom\cite{gog2015broom}抛弃GC,使用大小不同的region来管理具有相似生命周期的数据对象,
将大数据系统执行开销降低了34\%,但是需要开发者在代码中标记数据对象的生命周期,使用成本较高。Yak\cite{nguyen2016yak}是一个JVM的垃圾收集器,针对大数据中数据对象生命周期的epoch特性,将堆空间划分为一个个region,
每个region对应一个epoch,相同epoch的对象被分配到同一个region中,回收时也同时被回收。与JVM默认GC收集器 Parallel Scavenge相比,Yak将GC时延降低了1.4-44.3倍,Yak不需要对代码做很久修改,对开发者较为友好。

\section{数据传输}
在大数据系统中,一个很常见的任务是在集群里面的多个工作节点之间传输数据。例如在Hadoop上做一个MapReduce任务,在做reduce时,数据会从所有的map节点传输到reduce节点。
由于大数据系统是由托管式语言像Java或Scala实现的,所有的数据是以对象形式存储的。因此大数据系统中,数据从一个源节点传输到另一个目标节点需要经历三个步骤:
\begin{enumerate}
    \item 源节点将数据对象序列化为字节序列
    \item 源节点将序列化后的字节序列发送给目标节点
    \item 目标节点接收字节序列反序列化为数据对象
\end{enumerate}
尽管前人已经在序列化和反序列化方向上做了很多优化,但序列化与反序列化仍然是一个巨大的性能开销,占据了Spark 30\%的执行时间。因此Nguyen实现了Skyway\cite{nguyen2018skyway},
一种基于JVM的技术,直接在不同节点的堆上传输对象,避免了原来节点间传输数据需要序列化再反序列化所带来的巨大性能开销,开发者也因此不需要去手写序列化与反序列化函数。
实验表明,在Spark与Flink系统上,Skyway的性能要比Java序列化标准集(JSBS)中所有的Java序列化库(一共90个)都要好。Skyway使得Spark的性能提神了16\%~36\%,Flink提升了19\%。

\section{分布式}
分布式是让大数据任务更快速运行的一个重要方法。通过将数据集切割成多个子数据集,并在不同的节点上运行相同的任务处理子数据集可以大大提高数据处理的速度。
但是分布式也带来了其他问题,多节点各自独立工作不合作带来的重复劳动和性能损失。Lion和Chiu\cite{lion2016don}发现JVM多节点运行时,每个节点都需要warm-up
(加载类和解释字节码等),重复劳动并且带来了额外的性能开销,降低了可扩展性。所以他们实现了HotTub,通过在多个节点之间重用已经热身的JVM池,
消除了分布式大数据系统中每个节点都要Warm-up所带来的开销,提高了可扩展性。Maas\cite{maas2016taurus, maas2015trash}发现在延迟敏感的大数据系统中,一个节点的GC延迟
可能导致多个其他节点的等待。所以他实现了Taurus,一个包含所有节点运行时的整体性的运行时,用来协调多个节点做出配合的决定,例如GC。











% !TEX root = ../thesis.tex

\chapter{案例分析}
流行的数据处理框架比如Hadoop,Spark,Naiad或Hyracks都由托管式语言开发,例如Java,c\#,或Scala。JVM提供的内存自动管理——垃圾收集器(GC)为开发者提供便利。然而,垃圾收集开销是非常大的,GC在这些大数据系统中,可占据高达50\%的执行时间,严重损害了系统的性能。

GC执行缓慢的一个关键原因是大数据系统中的对象特征不匹配传统GC设计使用的启发式算法。在这一章,我们分析一个针对大数据系统中对象行为不匹配的解决方案——Yak。Yak能够智能地适应大数据系统的特性,因此,可以有效地管理在这些大数据系统中存在的大量对象。

我们首先在3.1节介绍Yak的研究思路,然后在3.2节分析该系统的局限性。


\section{研究思路}
\subsection{数据特征不匹配}

传统的分代GC都是基于“已存在时间越短的对象越容易被回收”的假设进行设计的。该假设适用于产生大量短暂临时数据的应用程序,对于这些应用来说, 传统分代GC是有效的——GC扫描一小部分堆就能够回收足够多的内存。但是,大数据应用程序的数据特征与传统分代GC假设不同,这种特征可以概括为两条路径,两种假设。

一个典型的大数据处理框架通常存在控制路径和数据路径的明显区别。控制路径有复杂的执行逻辑(例如,管理和调度集群,建立master node和worker node之间的通信);数据路径主要包括可以连接到一个数据处理管道的数据操作功能(例如,数据分区、Join或Aggregate,用户定义数据功能,例如Map或Reduce)。

这两条路径遵循不同的对象生命特征:控制路径上的对象数量小,存活时间短,生命周期符合分代特征,对它们使用传统的分代GC是合理的;而在数据路径上的对象,它们的生命周期呈现出完全不同的时域(epoch)特征(时域特征指的是一组对象在同一时间被创建,在经历一段较长的程序执行时间后才被同时回收的特征),所以不适合采用传统GC。
\begin{figure}[h]
    \centering
    \includegraphics[width=10cm,height=5cm]{figure/two_Path.png}
    \caption{大数据项目架构说明}
\end{figure}

这种不匹配会导致传统分代GC在数据密集型应用程序中的性能瓶颈。因为创建的对象寿命较长,GC的大部分精力都花费在跟踪巨大数量的对象,而不能及时回收内存。 

两条路径,两种特征启发了一种混合内存管理器——用两种不同的方式管理两种路径对象。对于具有时域特征的数据对象提出更加合适的内存管理方式,同一个时域创建的对象分配在同一个内存区域,同时进行内存回收。

\subsection{设计与实现}
\begin{figure}[h]
    \centering
    \includegraphics[width=12cm,height=12cm]{figure/layout.png}
    \caption{
        region的一个例子:(a)一个简单的程序被多个线程执行的过程,(b)对应产生的内存域结构
    }
    \label{img3}
\end{figure}
Yak的高层想法是基于明确区分的控制路径和数据路径,把堆分成一个小的控制空间(CS)和一个更大的数据空间(DS), 并使用不同的机制来管理这些空间。控制空间又被划分为新生代、老生代,采用传统的分代内存管理方法;数据空间则由多个内存域(region)组成,采用基于内存域的内存管理方法。
_

{\bfseries 问题1:何时创建和释放内存域?}

一个时域代表着一段程序的执行。当一个时域开始时,一个内存域被创建;当一个时域结束时,一个内存域被回收。一个内存域储存对应时域创建的所有对象。一个时域可以是用户自定义的,通过注释的方式实现。举例来说,在Hadoop中,一个用户定义的Map任务被setup()和cleanup()两个接口调用建立和关闭,开发者可以对这个两个调用的代码进行注释,从而定义一个时域。

为了建立统一的注释,Yak提供了一对用户注释,epoch\_tart() 和epoch\_end(),这两个注释在运行时被转换成两个本地函数调用,来提醒JVM一个时域的开始和结束。标注这些注释需要的手动工作开销是很小的。即使是新手,没有太多的系统知识,也可以很容易地找到并注释时域。Yak保证执行正确性,当然,时域标注的位置会影响性能:如果一个时域内的对象有不同的寿命,那这些对象会在时域结束后被复制,增加开销。

通常情况下,大数据框架中的系统调用已经具备了很好的时域特性,一旦创建了托管堆(在JVM的启动时期),会保留一系列虚拟地址作为DS。每个时域对应一个内存域,由一组固定长度的内存页组成。

在实践中,需要考虑关于时域概念的更多问题,其中一个是关于时域的嵌套关系。时域嵌套是指在执行的任何时刻,可能会有多个时域,为获得性能优势,Yak支持时域嵌套——内部时域不可达对象可在外部时域结束之前被回收,避免了内存增长过快而造成的内存占用。具体来说,如果一个正在运行的时域间执行epoch\_start(),一个新时域开始,创建一个新的内存域,成为上个时域的子时域。所有后续对象分配在子时域,直到遇到一个epoch\_end()。

另一个问题是,当多个线程同时执行数据处理代码的相同片段时,如何创建内存域。Yak为每个时域的动态实例创建内存域。当两个线程执行同一时域代码的相同片段时,他们获得自己的内存域而无需考虑同步问题。


\begin{figure}[h]
    \centering
    \includegraphics[width=12cm,height=7cm]{figure/epoch.png}
    \caption{
        region的一个例子:(a)一个简单的程序被多个线程执行的过程,(b)对应产生的内存域结构
    }
    \label{img2}
\end{figure}
在执行的任何时刻,可能会有多个时域,因此存在着多个内存域。基于嵌套关系,这些内存域是有一定顺序的,形成半格结构,如\ref {img2}所示,半格上的每个节点是<$r_{ij}$,$t_k$>形式的内存域,$r_{ij}$表示时期$r_i$的第$j$次执行,$t_k$表示线程。例如,区域<$r_{21}$,$t_1$>是<$r_{11}$,$t_1$>的子节点,因为在程序中,时域$r_2$是嵌套在$r_1$中的,并且执行线程相同,都是$t_1$。当被不同的线程执行时,两个时域(例如<$r_{11}$,$t_1$>和<$r_{12}$,$t_2$>)是并发的。

{\bfseries 第二个问题:如何正确并且高效地释放内存域?}

少量的对象存活时间可能比他们所处的时域存在时间长,这些存活对象必须在GC时被标识,并且迁移到安全的空间。Yak必须自动完成两项关键任务:(1)标识存活对象,(2)迁移存活对象。

对于第一项任务,Yak使用高效的算法来追踪跨内存域/空间的引用,并在运行时为每个内存域记录所有的传入引用。在内存域释放之前,Yak将这些引用作为根集合,以计算内存域中迁移对象的传递闭包。

对于第二项任务,对于每个转移对象O,Yak将O重定位到一个活内存域。为了实现这个目标,Yak标识每个传入的对O的跨内存域/跨空间引用的源内存域,并将其进行join操作,以在内存域半格上找到最小上界。例如,在图5中,对<$r_{21}$,$t_1$>和<$r_{11}$,$t_1$>进行join,返回<$r_{11}$,$t_1$>。若join了任意两个并发的区域,则返回到CS。直观地说,如果O有来自父区域和祖父区域的引用,则O应当被移动到祖父区域,如果O有来自两个不同的线程的引用,则必须被移动到CS。

在释放期间,如果其它线程访问将迁移对象的传递闭包,可能会导致不完全闭包。另外,当其他行程运行时,不能并发地移动对象,否则可能会引起数据竞争。为此,Yak使用轻量级的“stop-the-word”处理,以保证在释放时是线程安全的。当一个线程到达时域结束期,Yak暂停所有正在运行的线程,扫描它们的栈,计算释放内存域内所有潜在活对象的闭包。在其它的线程恢复前,将这些对象移动到各自的目标内存域。

Yak在Oracle的JVM OpenJDK 8 (build 25.0-b70)中实现。除了基于时域的技术,还修改了两个JIT编译器(C1和Opto),解释器,对象/堆布局,和Parallel Scavenge收集器,以管理CS。




\section{局限性}


\subsection{写障碍瓶颈}

\begin{figure}[h]
    \centering
    \includegraphics[width=12cm,height=7cm]{figure/algorithm1.jpg}
    \caption{
        算法1
    }
    \label{algorithm1}
\end{figure}
Yak需要有效地跟踪所有内存域/空间之间的引用。Yak通过三个步骤实现跟踪。首先,Yak将4字节字段(re)添加到每个对象的头部空间来记录对象所属的内存域信息。对象重新分配后,其头部字段更新为相应的内存域ID。CS用一个特殊的ID来表示。然后,Yak修改写障碍(例如,一小段代码增加引用a.f=b,会被写屏障捕获)来检测和记录基于堆的内存域/空间引用,修改后的算法如\ref{algorithm1}所示。

当对象a新增对对象b的引用,这个引用会被写障碍捕获,将引用关系存储到b所属内存域的记忆集(remember set)中去。

最后,当一个时域结束,Yak通过记忆集来追踪存活对象。

Yak通过写障碍来帮助追踪内存域/空间之间的引用,写障碍本身增加了开销。为了了解写屏障的开销,作者手动修改了GraphChi的执行引擎,强迫载荷滑动碎片和执行更新线程执行写障碍。这对线程序列化产生影响,使程序顺序化。对于GraphChi的三个项目,突变时间(例如non-pause时间)总体增长了24.5\%,这表明写障碍是主要的瓶颈.

现有的GC都通过手工编写/优化汇编代码来实现写障碍。Yak还没有相关的优化,我们希望能够通过优化汇编代码来降低写障碍的开销。


\subsection{适用性局限}
\begin{figure}[h]
    \centering
    \includegraphics[width=12cm,height=7cm]{figure/footprint1.jpg}
    \caption{
        GraphChi执行期间的内存足迹,横轴为执行时间(GC消耗73\%的运行时间)。(a)中的每个点表示一次GC后对内存的测量,(b)中每条线显示每次GC回收内存数量;垂直虚线表示时域界限。
    }
    \label{footprint1}
\end{figure}

Yak性能好坏,很大程度上取决于大数据框架的对象特征与假设中时域特征的契合程度,以下是GraphChi程序的运行足迹。

在GraphChi实验中,GC占据73\%的运行时间。每个时域持续约20秒(虚线表示)。如\ref{footprint1}所示。我们可以观察到时域结束点和内存下降存在明显的相关性。在每个时域执行期间,GC运行但只能回收很少的内存,是一种资源的浪费 (\ref{footprint1}(b))。

\begin{figure}[h]
    \centering
    \includegraphics[width=12cm,height=7cm]{figure/evaluation1.jpg}
    \caption{
        Yak与Parallel Scavenge 在GraphChi不同程序执行时间的对比。堆内存分配分别为6GB和8GB。
    }
    \label{evaluation1}
\end{figure}

因为GraphChi对象具有强烈的时域特征,Yak在GraphChi上性能优势明显,如\ref{evaluation1},对比Parallel Scavenge,Yak平均执行时间降低15\%。

作者在三种大数据框架——Hyracks、Hadoop、GraphChi都进行了实验,获得了显著的性能提升。

\begin{figure}[h]
    \centering
    \includegraphics[width=12cm,height=7cm]{figure/evaluation2.png}
    \caption{
        Spark执行LR期间的内存足迹,横轴为执行时间,(a)中的每个点表示一次GC后对内存的测量。
    }
    \label{evaluation2}
\end{figure}
但是, Yak对时域(region)的粗粒度划分导致Yak适用的局限性,只有严格满足对象时域特征的数据框架才能较好地发挥Yak优势,epoch假设不适用于Spark等内存计算系统。Spark LR执行的过程中,每一次GC均会回收较多的无用数据,\ref{evaluation2}是在spark中执行LR(Line Regression)内存的足迹,Yak对时域(region)的粗粒度划分导致Yak适用的局限性,只有严格满足对象时域特征的数据框架才能较好地发挥Yak优势,epoch假设不适用于Spark等内存计算系统。Spark的内存足迹没有显示出明显的时域界限,对象不满足严格的时域特征,Yak的内存回收方式不合适。

Yak粗粒度的时域划分导致系统性能与大数据框架中对象epoch特征紧密相关,当大数据框架中的对象epoch特征不明显时,Yak性能下降。

针对以上的缺陷,我们设想添加额外的、定义良好的细粒度时域来改善性能。

\subsection{低并行性}

Yak没有实现并行的原因就是在遇到多目的地问题时,必须严格按照拓扑排序迁移,所以整个迁移过程必须是串行的。

改进的思路是,我们可以在迁移前先进行完整的可达性分析,对含有相同对象的可达路径进行聚类,形成多个GC Task。Task内部可能存在多目的地问题,因此仍旧需要按照拓扑排序串行执行;而两个Task之间不存在交集,因此可以并行地执行对象迁移和引用更新,这样就可以提供一定的并行度,进一步优化GC性能。具体优化方案将在4讲解。


% !TEX root = ../thesis.tex

\chapter{优化方案}
内存域存在的明显的嵌套关系,所以在GC过程中,当遇到多目的地问题时,必须严格按照拓扑排序迁移,整个过程是串行的。接下来我们先分析Yak GC对象迁移的算法细节,接着提出我们提高迁移并行性的优化方案。

\section{算法分析}
\subsection{追踪引用}
当epoch\_end触发时,指向存活对象的引用可能驻留在3个地方——堆、本地栈、远程栈,接下来我们从细节分析Yak是如何追踪这三种引用的。

\textbf{(1)在堆中}  \ 如果一个对象$O_b$的引用被写入另一个对象$O_a$,并且$O_a$是在另一个内存域$r^{'}$上分配的,那么$O_b$的生命期比他所在的内存域$r$更长。算法1展示了写障碍捕捉了对$O_b$的域内引用。算法1检查了一个引用是否是跨内存域/空间的引用(第2行)。如果是,被引用所属内存域(REGION($O_a$))需要更新它的记忆集(第3行)。如果$O_a$和$O_b$在相同的内存域,包括CS(第1-2行),则不需要追踪该引用。
\begin{figure}[H]
    \centering
    \includegraphics[width=12cm,height=12cm]{figure/algorithm2.jpg}
    \caption{
        算法2: region回收算法
    }
    \label{algorithm2}
\end{figure}

(2)\textbf{在本地栈} \ 当一个对象被运行期外的栈变量引用时,会被迁移,如图\ref{algorithm2}所示。第3行创建的对象$b$的引用被分配给栈变量$a$。$a$在$b$所属时域结束后仍活着,$b$也存活。Yak在epoch\_end后,扫描本地栈,找到$r$中所有被活引用指向的对象$O_{var}$ (算法2中1-4行),Yak在前面加上一个代表$r$的父内存域($p$)的占位符,把这条记录加入$r$的记忆集。这条记录说明着当$r$被回收,$O_{var}$ 会被迁移到$p$。

(3)\textbf{在远程栈} \ 线程$t$创建的对象$O$可能会被线程$t^{'}$中的对象引用。例如图\ref{algorithm2},第4行创建的对象,在第10行中被另一个线程$t^{'}$引用并加载到栈中。为了避免读障碍对实用性和性能的影响,Yak通过“stop-the-world”的方法保证线程的安全性。当线程$t$回收内存域$r$时,Yak暂停所有其他的线程并且扫描它们的栈。被其他线程引用的$r$内对象会被标记。

\subsection{内存域释放}
算法2展示了在每个epoch\_end触发内存域释放算法。此算法计算存活对象的闭包,将对象迁移到其目标内存域后,回收整个内存域。

\textbf{寻找转移根}:一个内存域$r$有3种转移根。第一种是$r$的记忆集中所记录的指向内存域间/空间的引用,第二种是被本地栈所引用的对象,第三种被其它线程中的远程栈所引用的对象。

由于内存域间/空间的引用早已被写障碍所捕获,Yak只要在这阶段识别出栈中引用对象。Yak首先通过的本地栈标识存活的对象,如算法2的第1-4行所示。

接下来,Yak通过远程栈标识存活的对象。为此,Yak需要将线程同步(第5行)。当远程线程暂停时,Yak扫描其栈变量,并返回一个被这些变量所引用的位于$r$中的对象的集合。每个这样的对象(远程对象)都要被明确标记为根,并在计算传递闭包之前(第11行)移动到CS(第10行)。

直到完成闭包运算并将所有的存活对象移动到它们的目标区域,没有线程会被恢复。注意,即便远程线程的栈没有引用内存域$r$中的任何对象,让其继续运行也是不安全的。

所有的存活对象都会被重新分配,然后释放整个内存域$r$,$r$占用的所有页会被重新放回空闲页表(第13行)。



算法5-1显示了从所检测到的一组转移根上计算闭包的细节。由于所有其它线程都被暂停,所以闭包计算和对象迁移是同时完成的。闭包计算是基于记忆集$rs$和当前的释放区域$r$的。首先检查了记忆集$rs$(第1行):如果$rs$为空,则该内存域不包含迁移对象,因此可以被安全地释放。否则,需标识所有的可达对象并重定位。

Yak先检查记忆集中的每个引用${addr}$  $O_{b}$的有效性,再基于半格运算计算出每个转移根$O_{b}$要移动到的目标内存域(第2-4行),这些结果保存在map $promote$中。

然后,按照所有转移根的目标区域的拓扑顺序来遍历这些转移根(第5行的循环)。对于每个转移根$O_b$,在当前区域内执行宽度优先遍历,以识别$O_b$可访问的转移对象的传递闭包,将其放入队列$gray$中。在遍历中(第8-23行),计算每个(传递)转移对象的目标内存域并移动它们。

\begin{algorithm}[H]

    
    \nonumber
    \caption{Closure computation.}%Algorithm 3:Closure computation.
    \LinesNumbered %要求显示行号
    \KwIn{Remember Set $rs$ of Region $r$}%输入参数
    %\KwOut{output result}%输出
   % some description\; %\;用于换行

    \If{The remember set $rs$ of $r$ is NOT empty}{\Delta
      \ForEach{ Escaping root $O_b$  $\in$ $rs$ }{
      \ForEach{Reference addr $\xrightarrow{r'}$ ADDR($O_b$) in rs}{
      promote[$O_b$] ← JOIN (r′, promote[$O_b$])
    }
    }
    
    \ForEach{Escaping root $O_b$ in topological order of promote[$O_b$]}{
      
    Region tgt ← promote[$O_b$]
    
    Initialize queue gray with {Ob}
    
    \While{gray is NOT empty}{
    
    Object $O$ ← DEQUEUE($gray$)
    
    Write $tgt$ into the region field of $O$
    
    Object $O^∗$ ←MOVE($O$, $tgt$)
    
    Put a forward reference at ADDR($O$)
    
    \ForEach{Reference addr $\xrightarrow{x}$ ADDR(O) in r’s rs}{
    
    Write ADDR($O^∗$) into $addr$
    
    
      \If{$x \not= tgt$}{
    
        Add reference $addr$ $\xrightarrow{x}$ ADDR($O^∗$) into the remember set of region $tgt$
    
      }
    }
    
    \ForEach{Outgoing reference e of $O^∗$}{
      
    Object O′ ← TARGET($e$)
    
    \If{$O′$ is a forward reference}{
    
    Write the new address into $O^∗$
    
    }
    
    Region r′ ← REGION(O′)
    
    \uIf { $r′ = r$ } {
    
    ENQUEUE(O′, gray);
    
    } \ElseIf {$r′\not= tgt$}
    {
    Add reference ADDR($O$) $\xrightarrow{tgt}$ADDR($O′$) into the remember set of region $r′$
    
    }
    
    
    }
    
    }
    
    
    }
    
      }
    \nonumber
    \label{a31}
    \end{algorithm}

\subsection{更新记忆集并移动对象}

因为已经暂停了所有线程,对象的移动是安全的(第11行)。当对象$O$被移动了,需要更新存储了对它的引用的所有位置(栈和堆)。可能会有3种位置:(1)内存域内位置(即来自$r$中另一个对象的引用),(2)其它内存域或 CS,(3)栈。

\textbf{(1)内存域内位置}\ 为了处理内存域内引用,遵循 GC标准,在$O$的原始位置放置一个特殊前向引用(第12行)。这将对区域内作出通知:传入的引用的位置改变了——当$O$的位置被其它引用访问时,该前向引用会被用来更新O的引用源。

\textbf{(2)来自其它区域的对象}\ 对这些对象的引用必须被记录在$r$的记忆集中。因此,可以在记忆集中发现所有对$O$的内存域间引用,并用新地址$O^{'}$更新每个这样的引用源(第14行)。由于$O^{*}$属于新内存域$tgt$,最初进入内存域$r$的引用现在会进入内存域$tgt$内。如果包含这样引用的内存域不是$tgt$,则该引用需被明确添加到${tgt}$的记忆集合中(第16行)。

将$O$移动到内存域${tgt}$也许会导致新的内存域间/空间引用(第24-25行)。比如,如果对象$O^{*}$的目标域$r^{'}$不是$tgt$ (即$O^{'}$已经被其它转移根访问过了),需要在$r^{'}$的记忆集中添加新的条目 ${ADDR}$ $\xrightarrow{x}$ $O^{*}$。


\begin{figure}[H]
    \centering
    \includegraphics[width=12cm,height=7cm]{figure/snapshot.jpg}
    \caption{
        一个堆快照的例子:(a)是存活对象迁移前,(b)是迁移后。
    }
    \label{snapshot}
\end{figure}
\textbf{(3) 栈}\ 由于栈位置也被记录到记忆集中,对它们的更新方式与堆位置相同。例如,当$O$被移动了,第14行会在记忆集中更新每个对$O$的引用。如果$O$有栈引用(本地或远程),必须同样在记忆集中更新。

在传递闭包被计算和对象被迁移后,区域$r$的记忆集$rs$也就明确了(第26行)。图\ref{snapshot}显示了区域<$r_{21}$,$t_1$>被释放后的堆。对象 C、D和E是存活对象,将被移动到被计算好的目标内存域。由于D和E属于CS,在CS的记录集中添加了它们的传递引用2和3。对象F不可达,因此被自动释放。

以上就是Yak进行GC详细过程。我们注意到,在GC过程中stop-the-world是不可避免的。一是在寻找根对象时,扫描远程堆,通过stop-the-world保证线程安全,避免“危险对象移动”和“晃动对象”问题。二是在迁移对象时,避免了在其他线程正在访问将迁移对象的传递闭包,导致不完全闭包问题和当其他行程运行时,并发地移动对象,引起数据竞争问题。

stop-the-world意味着从应用中停下来并进入到GC执行过程中去。一旦stop-the-world发生,除了GC所需的线程外,其他线程都将停止工作,中断了的线程直到GC任务结束才继续它们的任务。GC调优通常就是为了改善stop-the-world的时间,我们希望能通过实现迁移对象这个过程的并行,降低迁移对象的时间,可以降低stop-the-world的时间,从而提高整个系统的性能。


\section{优化方案}

我们回到刚才的例子,Yak没有实现并行的原因就是在遇到多目的地问题时,必须严格按照拓扑排序先迁移f再迁移d,所以整个迁移过程必须是串行的。
\begin{figure}[H]
    \centering
    \includegraphics[width=12cm,height=5cm]{figure/n1.png}
    \caption{
        一个的多目的地迁移例子:e同时被d和f的引用,e应该迁移到层次更高的CS(f的迁移目的地)。
    }
    \label{snapshot}
\end{figure}
改进的思路是,我们可以在迁移前先进行完整的可达性分析,对含有相同对象的可达路径进行聚类,形成多个GC Task。Task内部可能存在多目的地问题,因此仍旧需要按照拓扑排序串行执行;而两个Task之间不存在交集,因此可以并行地执行对象迁移和引用更新,这样就可以提供一定的并行,进一步优化GC性能。如图\ref{snapshot},d,f,e为一个GC task;x,y,z为一个GC task;k,I,j,hl为一个GCtask。

\begin{figure}[H]
    \centering
    \includegraphics[width=12cm,height=7cm]{figure/n2.jpeg}
    \caption{
        GC tasks
    }
    \label{snapshot}
\end{figure}



\begin{algorithm}
    \caption{Closure computation.}%Algorithm 3:Closure computation.
    %\LinesNumbered %要求显示行号
    \KwIn{Remember Set $rs$ of Region $r$}%输入参数
    %\KwOut{output result}%输出
    % some description\; %\;用于换行
   % \begin{algorithmic}
    
    \If{The remember set $rs$ of $r$ is NOT empty}{\Delta
      \ForEach{ Escaping root $O_b$ $\in$ $rs$ }{
      \ForEach{Reference addr $\xrightarrow{r'}$ ADDR($O_b$) in rs}{
      promote[$O_b$] ← JOIN (r′, promote[$O_b$])
    }
    }
    
    Initialize map<int,set> tasks
    
    \ForEach{Escaping root $O_b$ in promote[$O_b$]}{
      
    Region tgt ← promote[$O_b$]
    Initialize queue gray with {Ob}
    
    Initialize task t\_new with {Ob}
    \While{gray is NOT empty}{
    Object $O$ ← DEQUEUE($gray$)
    
    t\_new.add(O)
    
    bool isTaskIncluded=false
    
    \ForEach{task t in tasks}{
    
    \If{t.contain(O)}{
    
    JOIN(t,t\_new)
    
    t\_new =t
    
    isTaskIncluded=true
    
    break
    }
    }
    \If{!isTaskIncluded}{
    tasks[GET\_NEW\_TASK\_ID()]=new\_t
    }
    }
    }
    
    
    
    
    
    
    
    
    
    \ForEach{task t in tasks}{
    \ForEach{Escaping root $O_b$ in topological order of task t}{
      
    Region tgt ← promote[$O_b$]
    Initialize queue gray with {Ob}
    \While{gray is NOT empty}{
    Object $O$ ← DEQUEUE($gray$)
    Write $tgt$ into the region field of $O$
    Object $O^∗$ ←MOVE($O$, $tgt$)
    Put a forward reference at ADDR($O$)
    \ForEach{Reference addr $\xrightarrow{x}$ ADDR(O) in r’s rs}{
    Write ADDR($O^∗$) into $addr$
      \If{$x \not= tgt$}{
        Add reference $addr$ $\xrightarrow{x}$ ADDR($O^∗$) into the remember set of region $tgt$
      }
    }
    
    
    
    \ForEach{Outgoing reference e of $O^∗$}{
      
    Object O′ ← TARGET($e$)
    \If{O′ is a forward reference}{
    Write the new address into $O^∗$
    }
    Region r′ ← REGION(O′)
    \uIf { r′ = r } {
    ENQUEUE(O′, gray);
    } \ElseIf {$r′\not= tgt$}
    {
    Add reference ADDR($O$) $\xrightarrow{tgt}$ADDR($O′$) into the remember set of region $r′$
    }
    }
    }
    }
    
    }
    }
    
    %\end{algorithmic}
    \end{algorithm}
   

改进的算法在5-2,1-4行依然是基于半格运算计算出每个转移根$O_b$要移动到的目标内存域,把这些结果保存在map $promote$中。第5行初始化一个map<int,set> $tasks$,用来存储分配在不同task的对象,6-19行对含有相同对象的可达路径进行聚类,放入一个task中。task拥有唯一的taskID标识,最终所有可达对象唯一存储在所属于task中。tasks存储所有task对象。

20-32行是算法并行性的实现。因为两个Task之间不存在交集,所以对每个task $t$,都有单独线程负责执行。这样通过实现迁移对象这个过程的并行,降低迁移对象的时间,可以降低stop-the-world的时间,从而提高整个系统的性能。




% !TEX root = ../thesis.tex

\chapter{评估与测试}
TODO

\chapter{总结}
本文对针对大数据场景的研究进行调研,发现高级语言虚拟机(JVM、CLR等)由于最初设计时并没有考虑大数据处理的特点,存在以下问题:
\begin{enumerate}
    \item GC时延较高
    \item 以对象形式存储数据导致的Memory Bloat
    \item Data shuffling开销较高
    \item 多节点调度带来的开销较高
\end{enumerate}

此外,本文具体分析了GC收集器Yak的研究思路,并指出Yak的局限性:写障碍瓶颈、适用性局限和低并行性。
最后,本文提出了一种Yak的优化方案,提高Yak的迁移并行性。



% 文后无编号部分
\backmatter

% 参考资料
\printbibliography[heading=bibintoc]

\end{document}