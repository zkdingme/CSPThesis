\chapter{总结}
本文对针对大数据场景的研究进行调研,发现高级语言虚拟机(JVM、CLR等)由于最初设计时并没有考虑大数据处理的特点,存在以下问题:
\begin{enumerate}
    \item GC时延较高
    \item 以对象形式存储数据导致的Memory Bloat
    \item Data shuffling开销较高
    \item 多节点调度带来的开销较高
\end{enumerate}

此外,本文具体分析了GC收集器Yak的研究思路,并指出Yak的局限性:写障碍瓶颈、适用性局限和低并行性。
最后,本文提出了一种Yak的优化方案,提高Yak的迁移并行性。

