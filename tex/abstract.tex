% !TEX root = ../thesis.tex

\begin{abstract}
  大数据时代带来了大数据项目的爆炸式发展。随着各个项目的发展与日益成熟,通过改进分布式计算框架本身来大幅提高性能的机会越来越少。越来越多的开发者将眼光投向了运行大数据框架的高级语言虚拟机本身,
  希望通过解决虚拟机本身的一些问题,提高分布式系统的性能和健壮性。
  由于这些高级语言虚拟机在最初设计时并没有充分考虑到大数据处理应用的特点,因此在性能方面往往会存在问题。近年来,有很多研究者从不同方向尝试解决这些问题。
  本文对大数据场景的研究进行调研,总结高级语言虚拟机存在的问题,具体分析垃圾收集器Yak的研究思路及其局限性,并针对Yak提出优化方案。
\end{abstract}
