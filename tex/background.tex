\chapter{背景}

\section{GC}
GC(垃圾收集)是托管式语言Java、Scala等与非托管式语言像C或C++的主要区别之一。
使用托管式语言进行开发,不需要专门编写内存分配与回收代码,也不需要去考虑
内存泄漏和溢出的问题。这确实给开发人员降低了不少工作压力,但也带来了两个新问题。
其一,GC时间开销太大,成为性能瓶颈,降低可扩展性。其二,由于内存管理都是托管的,
开发人员不能详细了解内存管理情况。这就导致一旦发生内存溢出与内存泄漏问题,开发人员
需要花更多的精力去解决。换言之,由于内存管理是托管的,内存管理的正确性与性能也都由
托管式语言对应的高级语言虚拟机(Java语言的JVM,.Net语言的CLR等)决定。所以一旦虚拟机的
垃圾回收策略与当前运行场景不适配(这常常发生,因为虚拟机的垃圾回收策略相对稳定,
大约2-3年才会更新一次,但是新场景层出不穷,所以往往策略是不能完全适配场景),那么内存管理的性能与正确性都会变得很糟糕。
GC的任务是解决以下三个问题:哪些内存需要回收、什么时候回收和如何回收。

\begin{figure}[h]
    \centering
    \includegraphics[width=12cm]{figure/JVM_memory_layout.pdf}
    \caption{JVM运行时数据区\cite{Understand_JVM}}
    \label{jvm_memory_layout}
\end{figure}

回答这三个问题首先需要了解虚拟机中的内存布局。以JVM为例,JVM在执行Java程序的过程中会把他所管理的内存划分为弱冠个不同的数据区域。
JVM运行时数据区如图\ref{jvm_memory_layout}所示。程序计数器存储正在执行的虚拟机字节码的地址。虚拟机栈包含局部变量表、操作数栈、动态链接、方法出口等信息。
其中局部变量表所需的内存空间在编译时完成分配,当进入一个方法时,这个方法需要在帧中分配多大的局部空间是完全确定的,不会改。
本地方法栈存储JNI调用。堆在虚拟机启动时创建,此内存区域唯一目的是存放对象实例。所有的对象实例和数据都要在堆上分配,
但是随着JIT编译器的发展与逃逸分析技术逐渐成熟,栈上替换、标量替换将会导致一些微妙的变化。方法区
存储被虚拟机加载的类信息、常量、静态变量、即时编译器编译后的代码等数据。其中方法区和堆是线程共享的,
也是JVM GC所针对的数据区。

哪些内存需要回收?托管式语言虚拟机中,数据是以对象的形式存储在堆中,对象由对象头、实例数据和对齐三部分组成。
其中对象头包含对象自身运行时的数据包括对象的hashcode、GC分代年龄等;实例数据包含对象的属性数据包括以指针形式存在的
指向其他对象的引用。GC主要针对的是堆中未被引用的对象。寻找未被引用对象是通过可达性分析算法实现的。可达性分析算法以
\texttt{GC Root}对象作为起点沿着对象间的引用链向下搜索,搜索不到的对象为不可达对象,GC时会被回收。\texttt{GC Root}
是GC被调用时一定存活的对象,JVM中\texttt{GC Root}由以下四种对象组成:
\begin{enumerate}
    \item 虚拟机栈(栈帧中的本地变量表)中引用的对象
    \item 方法区中类静态属性引用的对象
    \item 方法区中常量引用的对象
    \item 本地方法栈中JNI引用的对象
\end{enumerate}

在内存不够用时,虚拟机会调用GC。那么GC是如何运行的?虚拟机是通过不同的垃圾收集算法来执行GC的。包括标记-清除算法、
标记-整理算法、复制算法、分代收集算法和基于region的分代收集算法。标记-清除算法是指通过可达性分析标记出不可达对象后,直接清除。
标记-清除算法有两个不足:其一,清除之后会产生大量碎片,无法分配大对象;其二标记与清除两个过程都是针对整个堆,比较耗时。
复制算法将可用内存分为大小相等的两块,每次只使用其中一块,当一块内存用完时,将这块内存上还存活的对象复制到另一块内存上去,再将这一块内存全部清理掉。
复制算法解决了大量数据对象需要被清除时,清理耗时的问题,提高了效率,也解决了碎片化的问题。但是可用内存减小了,并且如果存活对象较多,复制也会比较耗时。
标记-整理算法将存活对象向顶端或低端移动,清除边界之外的对象,解决了标记-清除算法的碎片化问题,也较为高效,并且相对于复制算法,
标记-整理算法使用了全部的内存。

标记-清除算法、标记-整理算法和复制算法都是对整个堆进行标记和移动,当对象分配的越来越多时,标记和移动也越来越耗时。
但是,根据对Java应用的分析,数据对象的生命周期是不一样的。大部分对象的生命周期很短,小部分对象的生命周期很长。
分代收集算法就是基于这种观察结果被提出的。
分代收集算法将数据对象根据生命周期的不同分为新生代和老生代,并且将堆空间分为几块用来存放不同生命周期的对象,
分别应用最合适的垃圾收集算法。新生代生命周期较短,GC过后只有少量存活,所以使用复制算法。老生到对象存活率较高,
使用标记-清除或者标记-整理算法。对象被创建时,大对象直接进入老生代,小对象就在新生代进行分配,多次GC后任然
存活的对象会进入老生代。一般老生代会占据60\%以上的堆空间。显然新生代相较于老生代GC会比较频繁,新生代GC也只要对
新生代的做可达性分析,然后使用复制算法,老生代GC次数较少。因此分代收集算法提供了最高的收集效率。

\begin{figure}[h]
    \centering
    \includegraphics[width=12cm]{figure/region-based-gc.png}
    \caption{基于region的分代收集算法堆空间布局\cite{G1}}
    \label{jvm_region-gc}
\end{figure}
在相同条件下, 堆空间越大, 一次GC耗时就越长, 从而产生的停顿也越长。分代收集算法将堆空间划分为新生代与老生代两部分,
相较于之前,极大提高了效率。但是GC开销还是过大,现代服务器要求每次GC的时间要尽量短,最好控制在某个上限之下。因此
基于region的分代收集算法(又名分区分代收集算法)应运而生。如图\ref{jvm_region-gc}为了更好地控制GC产生的停顿时间, 基于region的分代收集算法将一块大的内存区域分割为多个小块, 
根据目标停顿时间, 每次合理地回收若干个小区间(而不是整个堆), 从而减少一次GC所产生的停顿。基于region的分代收集算法是目前
最新的垃圾收集算法,其性能也最忧。

\section{大数据特性}
与传统的面向对象程序相比,大数据应用有其独特的性质。研究\cite{gog2015broom,nguyen2015facade,bu2013bloat}显示,
一个典型的大数据程序通常存在控制路径和数据路径的明显区别。控制路径有复杂的执行逻辑(例如管理和调度集群、建立节点之间的通信、处理用户的交互请求像查询等);
数据路径主要包括可以连接到一个数据处理管道的数据操作功能(例如,数据分区、Join或Aggregate,用户定义数据功能,例如Map或Reduce)。

这两条路径有不同的堆使用模式。控制路径可能具有复杂的程序执行逻辑,但会创建对象较少。并且控制路径创建的对象的生命周期与
传统面向对象程序的对象相似:大部分对象生命周期较短。相比于控制路径,数据路径创造了大量的对象,占到对象总数的95\%\cite{bu2013bloat}。
并且数据路径上的数据处理代码需要被重复执行,称为\texttt{epochal behavior},每次执行称为一次\texttt{epoch}。每次\texttt{epoch}都会创建
大量的数据对象然后进行处理。在处理过程中,数据对象是全部存活的,处理结束之后,大部分数据对象可以被回收。这个处理过程通常不断。
这就导致数据路径创造对象的生命周期呈现出与完全不同的epoch特性。epoch特性是指一组对象在同一时间被创建,
在经历一段较长的程序执行时间后可以同时被回收。因此,大数据场景下,传统GC不适配,往往带来巨大的性能开销。



